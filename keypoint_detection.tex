\documentclass{comjnl}

\usepackage{amsmath}

\begin{document}


\title[Modelling Bidders in Sequential Automated Auctions]{3D Keypoint detection with Deep Neural Networks}
\author{El\'{i}as J. Puma}
\affiliation{Escuela Profesional de Ciencia de la Computaci\'{o}n,\\
Universidad Nacional de San Agust\'{i}n,\\
Arequipa, PE}
\email{eliasj.puma@gmail.com}

\shortauthors{E. Puma}

%\received{00 January 2009}
%\revised{00 Month 2009}


%\category{C.2}{Computer Communication Networks}{Computer Networks}
%\category{C.4}{Performance of Systems}{Analytical Models}
%\category{G.3}{Stochastic Processes}{Queueing Systems}
%\terms{Internet Technologies, E-Commerce}
\keywords{Keypoint detection; Deep Neural Networks; 3D Model; Sparse Autoencoders}


\begin{abstract}
3D keypoint detection plays a fundamental role in the Computer Vision field, detection of these salient points in the local surfaces of a 3D object is important in order to perform certain tasks such as registration, retrieval and simplification. There has been a lot of research in the field of 3D keypoint detection, most of them take a geometrical approach which have a good performance but lack flexibility to adapt to changes such as noise and high curvature points that are not keypoints to human preference. A good approach seems to be machine learning methods that can be trained with human annotated training data. In this paper a new method is proposed using deep neural network with sparse autoencoder as the regression model due to their great ability for feature processing. The analysis shows this method outperforms other methods that are widely used.  
\end{abstract}

\maketitle


\section{Introduction}
Several computer-dependent areas are benefited of the applications
that 3D Models have in them. The growth of 3D data has increased in
the latter years with the availability of low-cost 3D capture devices,
and the ability to analyse, proccess and select relevant information
from them is an active research area. 

3D interest point detection is a difficult task for several reasons. 
First, there are not any definitions for what a interest point is, 
most of the approaches consider the high level of protusion in a local
area as a keypoint characteristic. So, in planar sections of an area
vertices have a low interest level and in local areas with diferent
structures the interest level will be the opposite. Second, vertex 
density is different for every 3D model which makes harder the task of
selecting a local area. Third, information obtained from a 3D model
are only vertex positions and connectivity between them which means
the interest level will depend only from the information we can
retrieve from different calculations. These are not the only reasons
but are sufficient for explaining why this method is prepared to
handle these difficulties. 

In this paper we extend the work in~\cite{gelenbe06} to study the
outcome of strategies that a designated bidder may follow in an
English auction, in the presence of a collection of other bidders,
under the assumption that this ``special bidder'' (SB) observes
the parameters resulting from the auction as a collective (many
bidders and the seller) system. Note that in~\cite{gelenbe06}
bidders are lumped together in a pool, where everyone shares a
similar behaviour; whereas in this work we propose a
generalisation in that the SB is allowed to have its own activity
(bidding) rate which may differ from the other bidders', and
examine how the SB should select its bidding rate in a
self-serving manner.

We first sketch the model to be studied, and then in Section
\ref{Model} we analyse it in detail. The manner in which the model
provides performance measures of interest to the SB and to the
seller, is discussed in Section \ref{Performance} where we first
discuss how the SB can behave in order to optimise outcomes that
are in its best interest, and provide  numerical examples to
illustrate the approach and the model predictions. We then explore
how the SB can try to achieve balance and compete with the other
bidders in Section \ref{KeepUp}. Finally Section
\ref{Price-Dependent} generalises the analysis to the case where
the bidding rates depend on the current price attained in the
auction. Conclusions are drawn in Section \ref{Conclusions} where
we also suggest further work.


\section{Conclusions} \label{Conclusions}
In this paper we have considered auctions in which bidders make
offers that are sequentially increasing in value by a unit price
in order to minimally surpass the previous highest bid, and
modelled them as discrete state-space random processes in
continuous time. Analytical solutions are obtained and measures
that are of interest to the SB are derived.

The measures that can be computed in this way include the SB's
probability of winning the auction, its expected savings with
respect to the maximum sum it is willing to pay, and the average
time that the SB spends before it can make a purchase. An
extension of the model that incorporates price-dependent
behaviours of the agents has also been presented.

The model allows us to quantitatively characterise intuitive and
useful trade-offs between improving the SB's chances of buying a
good quickly, and the price that it has to pay, in the presence of
different levels of competition from the other bidders.


There are interesting extensions and applications of these models
that can be considered, such as the behaviour of bidders and
sellers that may have time constraints for making a purchase, and
the possibility of the SB's moving among different auctions so as
to optimise measures which represent its self-interest. Another
interesting area of study may be to examine bidders who are
``rich'' and are willing to drive away rivals at any cost, and who
may create different auction environments for bidders that have
significantly different levels of wealth. Yet another area of
interest concerns auctions where items are sold in batches of
varying sizes, with prices which depend on the number of items
that are being bought.

\ack{This research was undertaken as part of the ALADDIN
(Autonomous Learning Agents for Decentralised Data and Information
Networks) project and is jointly funded by a BAE Systems and EPSRC
(Engineering and Physical Research Council) strategic partnership
(EP/C548051/1).}


\nocite{*}

\bibliographystyle{compj}
% \bibliography{ModellingBidders}
\input{sample.bbl}


\end{document}
