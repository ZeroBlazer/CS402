\documentclass{comjnl}

\usepackage{amsmath}

\begin{document}


\title[Modelling Bidders in Sequential Automated Auctions]{3D Keypoint detection with Deep Neural Networks}
\author{El\'{i}as J. Puma}
\affiliation{Escuela Profesional de Ciencia de la Computaci\'{o}n,\\ Universidad Nacional de San Agust\'{i}n,\\ Arequipa, PE} \email{eliasj.puma@gmail.com}

\shortauthors{K. Velan}

%\received{00 January 2009}
%\revised{00 Month 2009}


%\category{C.2}{Computer Communication Networks}{Computer Networks}
%\category{C.4}{Performance of Systems}{Analytical Models}
%\category{G.3}{Stochastic Processes}{Queueing Systems}
%\terms{Internet Technologies, E-Commerce}
\keywords{Keypoint detection; Deep Neural Networks; 3D Model; Sparse Autoencoders}


\begin{abstract}
3D keypoint detection plays a fundamental role in the Computer Vision field, detection of these salient points in the local surfaces of a 3D object is important in order to perform certain tasks such as registration, retrieval and simplification. There has been a lot of research in the field of 3D keypoint detection, most of them take a geometrical approach which have a good performance but lack flexibility to adapt to changes such as noise and high curvature points that are not keypoints to human preference. A good approach seems to be machine learning methods that can be trained with human annotated training data. In this paper a new method is proposed using deep neural network with sparse autoencoder as the regression model due to their great ability for feature processing. The analysis shows this method outperforms other methods that are widely used.  
\end{abstract}

\maketitle


\section{Introduction}



Computer systems and the Internet have enabled a wide variety of
automated trading schemes which are in use for stocks,
commodities, derivatives and other financial instruments. More
recently, the Internet has also allowed individuals to buy and
sell various items in an open and easily accessible manner. Thus
we can envisage a future when large parts of the economy will be
driven by sequences of automated and interconnected trading
patterns.

Auctions are a convenient mechanism for formalising the rules with
which such automating trading schemes can be conducted, and they
have been widely used for many centuries in human based trading
and commerce. In recent work the stochastic behaviour of
collections of bidders acting in an auction has been analysed
through the use of discrete state-space and continuous time
probability models~\cite{gelenbe06}. The approach constructs
stochastic dynamical models of auctions and bidders, and then
obtains the steady-state behaviour to compute significant
properties of both the ``one-shot'' outcome, and the long-run
repetitive outcome of auctions. One-shot properties include the
probability distribution and the expected sale price, while
long-term properties include the income per unit time obtained by
the seller over a large number of transactions.
In~\cite{gelenbe_inPress} the model has been extended to study a
network of interconnected auctions, where bidders are allowed to
move freely between auctions, and an analytical solution has been
obtained. These mathematical techniques, and their variants, have
also been successfully deployed in other diverse range of problem
areas: from biological applications in modelling populations of
viruses and agents~\cite{Gelenbe-SoftwareViruses}, to
communication systems in minimising packet travel time across a
wireless network~\cite{Gelenbe07_DiffModelPacketTravel,
Gelenbe_06_Workshop_TravelDelay}, and choosing adaptive routing
decisions~\cite{Gelenbe03_Sensible}.

In reality, knowledge of the existence of opportunity and
availability to purchase similar goods in the future may influence
a bidder to choose to forgo the opportunity of procuring a good at
a high cost, and instead wait in the hope of securing a better
deal later. We can also imagine situations where goods are
reusable resources that are not sold per se, but rented out and
returned to the seller at the end of some period, so that bidder
agents who are not in urgent need to obtain a good immediately may
defer a purchase. Interesting work along these lines, where
bidders exhibit rational forward-looking behaviour when deciding
strategies for the current auction can be found in
\cite{zeithammer06}. A possible application for auctions of
reusable goods in allocating computing resources is given
in~\cite{gagliano_etal_95}.

When we discuss automated auctions, we will invariably imagine
software agents representing human counterparts in the digital
marketplace~\cite{maes99}, acting autonomously and yet guided by
its design objectives to fulfill the interests of their owners to
the best possible extent. In such instances it is crucial that the
underlying communications infrastructure is designed to allow and
support the user, i.e. the agent, to set the criteria for its
service requirements. For example, a bidder agent may have certain
specific needs with respect to its connectivity to the seller, and
so may request for a network path with the minimum overall delay
to the seller be established, or an agent physically located on a
mobile node may prefer a connection that consumes the minimum
power, or any weighted combination of its other goals. In this
regard, auctions, or any digital marketplace activity for that
matter, will benefit from autonomic
communications~\cite{Dobson_etal_06} where emphasis is placed to
insulate the user experience and services from changes, whether
predicted or not, to the underlying infrastructure. In particular,
a self-aware network, as proposed and implemented in the
\emph{cognitive packet network}~(CPN)~\cite{Gelenbe_etal_Nagoya,
Gelenbe_Gellman_Su_ISCC03, Gelenbe-Gellman-Lent-Liu04},
accomplishes this by online internal probing and measurement
mechanisms that are used for self-management, so that it adapts
itself to provide the user the best effort quality of service
(QoS). The CPN performs these corrective actions by using random
neural networks with reinforcement
learning~\cite{Gelenbe_01_CompNetw_CPN, Gelenbe_Mascots02,
Gelenbe_Lent_Nunez_04}, and finds newer routes with improved QoS
using genetic algorithms~\cite{Gelenbe_Liu06_workshop,
Gelenbe_Liu06_GeneticAlgorithmsRouteDiscovery}. In a wireless
mobile ad hoc network environment, where power efficiency is an
overriding concern, the CPN has been extended to incorporate
power-awareness~\cite{Gelenbe-Lent_2004}, and the problem of
controlling the admission of new users into the network while
preserving the QoS of all users has been addressed in
\cite{gelenbe_sakellari_08}.

Beyond ensuring QoS for users, at a different level of
abstraction, the ongoing research in autonomic communications has
a broader objective of providing an intelligent platform for
efficient interaction between digital objects such as users and
services~\cite{Gelenbe05_UsersServicesIntNetworks}, or as in our
context between buyer and seller agents. An important aspect of
applying the ``intelligence'', as discussed in
\cite{ferdinando_etal08}, is in creating new knowledge based on
collected raw sensitive data, through a knowledge network which,
in turn, can be used to enhance economic efficiency. In this
example, it is shown that a good allocation efficiency can be
achieved in a trading application where the aggregated knowledge
is used to create new markets so that sellers can respond to
buyers' needs as they arise.


Security is another vital feature in a communication network,
especially when transactions involving e-commerce activities such
as auctions are conducted; the servers running these applications
are easy targets for attackers, either as a malicious act of
sabotage or for profitable gains. The denial of service (DoS)
attack is a particularly critical threat, since it is easy to
launch, and by its usually distributed nature, difficult to
protect against. Thus an autonomic approach to defending the
network, based on self-monitoring and adaptive measures, has been
suggested in \cite{Gelenbe_Loukas07_SelfAwareDoS}, and several
biologically inspired DoS detectors have been evaluated in
\cite{Loukas_Oke_07, Loukas_Oke_LikelihoodRatios,Oke_etal07}.


In this paper we extend the work in~\cite{gelenbe06} to study the
outcome of strategies that a designated bidder may follow in an
English auction, in the presence of a collection of other bidders,
under the assumption that this ``special bidder'' (SB) observes
the parameters resulting from the auction as a collective (many
bidders and the seller) system. Note that in~\cite{gelenbe06}
bidders are lumped together in a pool, where everyone shares a
similar behaviour; whereas in this work we propose a
generalisation in that the SB is allowed to have its own activity
(bidding) rate which may differ from the other bidders', and
examine how the SB should select its bidding rate in a
self-serving manner.

We first sketch the model to be studied, and then in Section
\ref{Model} we analyse it in detail. The manner in which the model
provides performance measures of interest to the SB and to the
seller, is discussed in Section \ref{Performance} where we first
discuss how the SB can behave in order to optimise outcomes that
are in its best interest, and provide  numerical examples to
illustrate the approach and the model predictions. We then explore
how the SB can try to achieve balance and compete with the other
bidders in Section \ref{KeepUp}. Finally Section
\ref{Price-Dependent} generalises the analysis to the case where
the bidding rates depend on the current price attained in the
auction. Conclusions are drawn in Section \ref{Conclusions} where
we also suggest further work.


\section{Conclusions} \label{Conclusions}
In this paper we have considered auctions in which bidders make
offers that are sequentially increasing in value by a unit price
in order to minimally surpass the previous highest bid, and
modelled them as discrete state-space random processes in
continuous time. Analytical solutions are obtained and measures
that are of interest to the SB are derived.

The measures that can be computed in this way include the SB's
probability of winning the auction, its expected savings with
respect to the maximum sum it is willing to pay, and the average
time that the SB spends before it can make a purchase. An
extension of the model that incorporates price-dependent
behaviours of the agents has also been presented.

The model allows us to quantitatively characterise intuitive and
useful trade-offs between improving the SB's chances of buying a
good quickly, and the price that it has to pay, in the presence of
different levels of competition from the other bidders.


There are interesting extensions and applications of these models
that can be considered, such as the behaviour of bidders and
sellers that may have time constraints for making a purchase, and
the possibility of the SB's moving among different auctions so as
to optimise measures which represent its self-interest. Another
interesting area of study may be to examine bidders who are
``rich'' and are willing to drive away rivals at any cost, and who
may create different auction environments for bidders that have
significantly different levels of wealth. Yet another area of
interest concerns auctions where items are sold in batches of
varying sizes, with prices which depend on the number of items
that are being bought.

\ack{This research was undertaken as part of the ALADDIN
(Autonomous Learning Agents for Decentralised Data and Information
Networks) project and is jointly funded by a BAE Systems and EPSRC
(Engineering and Physical Research Council) strategic partnership
(EP/C548051/1).}


\nocite{*}

\bibliographystyle{compj}
% \bibliography{ModellingBidders}
\input{sample.bbl}


\end{document}
